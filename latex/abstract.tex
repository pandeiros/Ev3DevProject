% Streszczenie w języku Polskim

\setlength{\absleftindent}{0cm}
\setlength{\absrightindent}{0cm}
\renewcommand{\abstractnamefont}{\normalfont\Large\bfseries}
\begin{abstract}
    \large
    \indent Celem niniejszej pracy jest przetestowanie programowalnych możliwości LEGO EV3 i jego zastosowań w sterowaniu robotem mobilnym. Zaprojektowana aplikacja jest napisana w języku C++, z wykorzystaniem biblioteki ev3dev. Jej główną zaimplementowaną funkcjonalnością jest automat sterujący zachowaniami robota oraz reagujący na zdarzenia generowane przez bodźce z otoczenia. W pracy opisano również system komunikacji między poszczególnymi agentami za pomocą sieci bezprzewodowej oraz przedstawiono działanie jednostki nadzorującej cały system.\\

    \noindent
    Słowa kluczowe: LEGO Mindstorms EV3, agent mobilny, zachowanie.\\
\end{abstract}

~\\[0.1cm]
% Linia pozioma
\noindent \rule{\linewidth}{0.2mm}\\

% Streszczenie w języku angielskim
\begin{otherlanguage}{english}
\renewcommand{\abstractname}{}    % clear the title
\renewcommand{\absnamepos}{empty} % originally center

\begin{abstract}
    ~\\[0.05cm]
    \begin{center}
        {\Large Capabilities of programming the LEGO EV3.}
    \end{center}
    ~\\[0.1cm]
    \large
    The goal of this thesis is to test the capabilities of programming the LEGO EV3 brick and its use in controlling a mobile robot. The designed application is written using C++ language and ev3dev library. The main implemented feature is the state machine that supervises robot's behaviour and reacts to different events. This work also covers details about wireless communication system between agents and describes the central unit which takes care of the whole system.\\

    \noindent
    Keywords: LEGO Mindstorms EV3, mobile agent, behaviour
\end{abstract}
\end{otherlanguage}
