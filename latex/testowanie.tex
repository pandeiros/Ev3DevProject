\chapter{Testowanie aplikacji}
\label{ch:testowanie}

\section{Testy poprawności}

\subsection{Testowanie pojedynczych akcji}

\begin{description}
    \item[Ruch z zadaną prędkością -]wykonanie akcji powoduje utrzymywanie stałej prędkości obu motorów przez czas nieograniczony
    \item[Zatrzymywanie -]robot wprawiony w ruch zatrzymuje się niemal natychmiastowo. W zależności od sposobu zatrzymywania się, czas po którym prędkość jest równa zeru, a co za tym idzie, dystans hamowania, zmienia się. W\,przypadku trybu {\tt coast} jest on największy i równy od 1 do 5 cm w zależności od szybkości jazdy. Zaś w przypadku trybu {\tt hold}, koła zatrzymują się natychmiast i dystans maleje do maksymalnie 2 cm na śliskiej powierzchni
    \item[Ruch na zadaną odległość -]akcja wykonywana poprawnie z dokładnością do\,kilku nadmiarowych centymetrów. Nadwyżka odległości zależy od ustalonego trybu zatrzymywania oraz rodzaju nawierzchni, a wynika z opóźnienia odczytów z efektorów. Na podstawie zmierzonego promienia koła, zadanej odległości oraz ilości jednostek w jednym pełnym obrocie, można wyliczyć docelową pozycję motoru i zatrzymać go, gdy ta zostanie osiągnięta
    \item[Obrót -]jest wykonywany wokół osi równo oddalonej od obu kół. Podobnie jak w przypadku ruchu, sprawdzana jest zmiana położenia motoru, której docelowa wartość wyliczana jest na podstawie rozstawu kół i kąta. Gdy jeden efektor ma zadaną wartość dodatnią prędkości, drugi ma wartość przeciwną. Wykryto odchylenie o kilka stopni przy większych prędkościach. Zalecana jest zatem mała prędkość obrotu, w celu uzyskania większej precyzji.
\end{description}

\subsection{Testowanie obsługi zdarzeń}

\subsubsection{\underline{Rozpoczęcie przetwarzania zachowania}}


\begin{description}
    \item[Warunki:]Robot implementuje wymagane przez zachowanie akcje, posiada urządzenia umożliwiające wykonanie tych akcji oraz jest w stanie ACTIVE, WORKING lub PAUSED
    \item[Przypadek 1 - Brak wymaganych urządzeń:]Robot zgłasza błąd i przerywa działanie
        \lstinputlisting{listings/test1-1}

    \item[Przypadek 2 - Robot nie implementuje akcji:]Zgłoszenie ostrzeżenia (domyślnie wysyłany jest komunikat zwrotny o niepowodzeniu przypisania zachowania)
        \lstinputlisting{listings/test1-2}

    \item[Przypadek 3 - Nieprawidłowy stan:]Brak działania (domyślnie dostarczono zły komunikat, więc jest on ignorowany)
        \lstinputlisting{listings/test1-3}

    \item[Przypadek 4 - Warunki spełnione:]Następuje przypisanie zachowania, a\,następnie wygenerowanie zdarzenia BEHAVIOUR\_START, które przechwycone przez klasę {\tt Robot} uruchamia je.
        \lstinputlisting{listings/test1-4}
\end{description}

\subsubsection{\underline{Ostrzeżenie o przeszkodzie}}

\begin{description}
    \item[Warunki:]Robot posiada sensor nacisku oraz zachowanie ma zdefiniowaną reakcję na zdarzenie OBSTACLE\_DETECTED
    \item[Przypadek 1 - Brak sensora:]Jeśli robot powinien go posiadać, zgłaszany jest błąd. Jeśli nie - to zdarzenie nie będzie generowane
        \lstinputlisting{listings/test2-1}
    \item[Przypadek 2 - Brak zdefiniowanej reakcji:]Zdarzenie jest wygenerowane, ale nie podjęte są żadne działania
        \lstinputlisting{listings/test2-2}
    \item[Przypadek 3 - Warunki spełnione:]Zgłoszone zdarzenie jest przesłane do\,zachowania i następuje natychmiastowe zatrzymanie robota, a następnie przejście w stan pośredni oraz kontynuowanie zachowania w kolejnym domyślnym stanie.\\
        \lstinputlisting{listings/test2-3}
\end{description}

\subsubsection*{\underline{Ostrzeżenie o bliskim obiekcie}}

\begin{description}
    \item[Warunki:]Robot posiada sensor odległości oraz zachowanie ma zdefiniowaną reakcję na zdarzenie PROXIMITY\_ALERT
    \item[Przypadek 1 - Brak sensora:]Jeśli robot powinien go posiadać, zgłaszany jest błąd. Jeśli nie - to zdarzenie nie będzie generowane
        \lstinputlisting{listings/test3-1}
    \item[Przypadek 2 - Brak zdefiniowanej reakcji:]Zdarzenie jest wygenerowane, ale nie podjęte są żadne działania
        \lstinputlisting{listings/test3-2}
    \item[Przypadek 3 - Warunki spełnione:]Zgłoszone zdarzenie jest przesłane do\,zachowania i następuje natychmiastowe zatrzymanie robota, a następnie przejście w stan pośredni oraz kontynuowanie zachowania w kolejnym domyślnym stanie.\\
        \lstinputlisting{listings/test3-3}
\end{description}

\subsubsection{\underline{Niski poziom baterii}}

Przetestowany dla poziomu większego niż minimalne dopuszczalne napięcie. Po\,jego przekroczeniu, wygenerowane zostało zdarzenie BATTERY\_LOW, wysłano komunikat o niskim naładowaniu baterii oraz nastąpiło zamknięcie aplikacji.

\subsection{Testowanie przejść między stanami}

Komunikaty niezarządzające przetwarzaniem danego stanu są ignorowane.

\subsubsection{\underline{Stan IDLE}}

\begin{description}
    \item[Początek stanu -]Dioda miga na żółto
    \item[Brak odpowiedzi od nadzorcy -]Po ustalonym czasie, przejście w stan PANIC
        \lstinputlisting{listings/test6-idle-1}
    \item[Synchronizacja z nadzorcą -]Nadanie identyfikatora agentowi i przejście w\,stan ACTIVE
        \lstinputlisting{listings/test6-idle-2}
\end{description}

\subsubsection{\underline{Stan ACTIVE}}

\begin{description}
    \item[Początek stanu -]Dioda miga na zielono
    \item[Brak odpowiedzi od nadzorcy -]Po ustalonym czasie, przejście w stan PANIC
        \lstinputlisting{listings/test6-active-1}
    \item[Przesłanie zachowania i komunikatu START -]Przypisanie zachowania, przejście w stan WORKING, rozpoczęcie przetwarzania zachowania.
        \lstinputlisting{listings/test6-active-2}
\end{description}

\subsubsection{\underline{Stan WORKING}}

\begin{description}
    \item[Początek stanu -]Dioda świeci stale na zielono
    \item[Brak odpowiedzi od nadzorcy -]Po ustalonym czasie, przejście w stan PAUSED oraz zatrzymanie zachowania
        \lstinputlisting{listings/test6-working-1}
    \item[Przesłanie komunikatu STOP -]Zatrzymanie zachowania, przejście do\,stanu ACTIVE
        \lstinputlisting{listings/test6-working-2}
    \item[Przesłanie komunikatu PAUSE -]Zatrzymanie zachowania, przejście do\,stanu PAUSED
        \lstinputlisting{listings/test6-working-3}
\end{description}

\subsubsection{\underline{Stan PAUSED}}

\begin{description}
    \item[Początek stanu -]Dioda świeci stale na pomarańczowo
    \item[Brak odpowiedzi od nadzorcy -]Po ustalonym czasie, przejście w stan PANIC
        \lstinputlisting{listings/test6-paused-1}
    \item[Przesłanie komunikatu RESUME -]Wznowienie zachowania, przejście do\,stanu WORKING
        \lstinputlisting{listings/test6-paused-2}
    \item[Przesłanie komunikatu STOP -]Zatrzymanie zachowania, przejście do\,stanu ACTIVE
        \lstinputlisting{listings/test6-paused-3}
\end{description}

\section{Testy funkcjonalne}

\subsection{Testowanie zachowania złożonego i komunikacji}

Na poniższym listingu przedstawiono jeden pełny przebieg zachowania wraz z komunikatami komunikacyjnymi i przejściami między stanami.

\lstinputlisting{listings/test-explore}
