\chapter{Wstęp}
\label{ch:wstep}

Celem niniejszej pracy było zaprojektowanie i implementacja aplikacji kontrolującej zachowanie robotów mobilnych. Jest odpowiedzialna za nadzór nad zachowaniami robotów (agentów), sterowanie sensorami i efektorami oraz komunikację z wyższymi warstwami architektury, weryfikującymi poprawność całego systemu.

...

% TODO

\section{Motywacja}

Motywacją do podjęcia powyższego tematu było przetestowanie możliwości programowalnych oraz technicznych robota zbudowanego z klocków LEGO Mindstorms EV3. Aplikacja działająca na urządzeniu była napisana z użyciem biblioteki ev3dev \cite{ev3dev} w języku C++.

Możliwość dostępu do sterującego klockiem centralnym systemu Linux zdejmuje ograniczenie używania prostych środowisk graficznych i pozwala osiągnąć dużo więcej małym kosztem. Należało zatem sprawdzić, co najnowsza wersja LEGO Mindstorms ma do zaoferowania, w szczególności:
\begin{itemize}
    \item Wydajność napisanych aplikacji z użyciem ww. bilbioteki.
    \item Skuteczność komunikacji z wykorzystaniem bezprzewodowej sieci Wi-Fi.
    \item \dots
    % TODO
\end{itemize}

\section{Założenia}

Zostały przyjęte następujące założenia:
\begin{itemize}
  \item Każdy agent jest zdolny do wykonywania pewnych konkretnych zachowań niezależnie od pozostałych agentów.
  \item Zachowania te są reprezentowane za pomocą automatu skończonego.
  \item Dany robot może, ale nie musi być zdolny do wykonania konkretnej czynności. Jest to zależne od podłączonych do niego sensorów i efektorów.
  \item Każdy robot samodzielnie generuje sposób wykonania danej akcji, na podstawie dostępnych urządzeń.
  \item Zachowania mogą być dynamicznie tworzone z użyciem specjalnej składni.
  \item Agenci komunikują się zarówno z jednostką centralną (nadzorująca) jak i między sobą za pomocą sieci bezprzewodowej.
  \item Urządzeniem nadzorującym może być inny robot, ale pożądana jest też możliwość kontroli z poziomu zwykłego komputera.
  \item System powinien dostosować się do braków w łączności, umożliwiając komunikację przez pośrednictwo innych agentów.
  \item Agent może poruszać się tylko po płaskiej powierzchni.
\end{itemize}

\section{Ograniczenia}

\ldots
% TODO

\section{Zawartość rozdziałów}

\begin{description}
  \item[ {~\ref{ch:wstep} \hyperref[ch:wstep]{Wstęp} -} ]zawiera cel pracy razem z założeniami i ograniczeniami.
  \item[ {~\ref{ch:opis_systemu} \hyperref[ch:opis_systemu]{Opis systemu} -} ]
  \item[ {~\ref{ch:budowa_aplikacji} \hyperref[ch:budowa_aplikacji]{Budowa aplikacji} -} ]
  \item[ {~\ref{ch:zachowania} \hyperref[ch:zachowania]{Zachowania} -} ]
  \item[ {~\ref{ch:komunikacja} \hyperref[ch:komunikacja]{Komunikacja} -} ]
  \item[ {~\ref{ch:testowanie} \hyperref[ch:testowanie]{Testowanie aplikacji} -} ]
\end{description}

% TODO
