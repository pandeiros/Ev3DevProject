\chapter{Wstęp}
\label{ch:wstep}

Celem niniejszej pracy było zaprojektowanie i implementacja aplikacji kontrolującej zachowanie robotów mobilnych. Jest ona odpowiedzialna za nadzór nad\,zachowaniami robotów (agentów), sterowanie \glslink{sensor}{sensorami} i efektorami oraz komunikację z wyższymi warstwami architektury, weryfikującymi poprawność całego systemu. Zachowania robota są opisane za pomocą automatów skończonych. Warstwa komunikacyjna dostarcza interfejs przesyłania specjalnych komunikatów za pomocą protokołu \Gls{udp}.\\

Aplikacja jest zaprojektowana do uruchamiania na systemach Unixowych, w\,szczególności na klocku centralnym LEGO, a także na komputerze osobistym w\,roli nadzorcy. Warstwa połączeniowa jest punktem wspólnym architektury \Gls{arm} z inną, używając do komunikacji interfejsu gniazd sieciowych.
\clearpage

\section{Motywacja}

Motywacją do podjęcia powyższego tematu było przetestowanie możliwości technicznych oraz programowania robota zbudowanego z klocków LEGO Mindstorms EV3. Aplikacja działająca na klocku była napisana z użyciem biblioteki ev3dev \cite{ev3dev} w języku C++.

Możliwość dostępu do sterującego klockiem centralnym systemu Linux zdejmuje ograniczenie używania prostych środowisk graficznych i pozwala osiągnąć dużo więcej małym kosztem. Należało zatem sprawdzić, co najnowsza wersja LEGO Mindstorms ma do zaoferowania, w\,szczególności:
\begin{itemize}
    \item Wydajność napisanych aplikacji z użyciem ww. bilbioteki
    \item Skuteczność komunikacji z wykorzystaniem bezprzewodowej sieci Wi-Fi
    \item Dokładność dostarczanych odczytów z sensorów oraz szybkość i niezawodność reakcji na nietypowe zdarzenia.
\end{itemize}

\section{Założenia}

Zostały przyjęte następujące założenia:
\begin{itemize}
  \item Każdy agent jest zdolny do wykonywania pewnych konkretnych zachowań niezależnie od pozostałych agentów
  \item Zachowania te są reprezentowane za pomocą automatu skończonego
  \item Dany robot może, ale nie musi być zdolny do wykonania konkretnej czynności. Jest to zależne od podłączonych do niego sensorów i \glslink{effector}{efektorów}
  \item Każdy robot samodzielnie generuje sposób wykonania danej akcji na podstawie dostępnych urządzeń
  \item Zachowania mogą być dynamicznie tworzone z użyciem specjalnej składni
  \item Urządzeniem nadzorującym może być inny robot, ale pożądana jest też możliwość kontroli z poziomu zwykłego komputera
  \item Agenci komunikują się zarówno z jednostką centralną (nadzorująca), jak\,i między sobą za pomocą sieci bezprzewodowej
  \item Do poprawnej komunikacji wymagana jest jedna, wspólna dla wszystkich jednostek, istniejąca sieć
  \item System powinien dostosować się do braków w łączności, umożliwiając komunikację za pośrednictwem innych agentów
  \item Agent może poruszać się tylko po płaskiej powierzchni.
\end{itemize}

% \section{Ograniczenia}
% TODO

\section{Zawartość rozdziałów}

\begin{description}
  \item[ {~\ref{ch:wstep}. \hyperref[ch:wstep]{Wstęp} -} ]zawiera cel pracy razem z założeniami i ograniczeniami
  \item[ {~\ref{ch:opis_systemu}. \hyperref[ch:opis_systemu]{Opis systemu} -} ]opisuje dokładnie aspekty projektowe i technologiczne oraz zawiera wytyczne dotyczące wykorzystanych narzędzi
  \item[ {~\ref{ch:budowa_aplikacji}. \hyperref[ch:budowa_aplikacji]{Budowa aplikacji} -} ]szczegółowo omawia wykorzystane w projekcie klasy, ich wzajemne zależności oraz podział na moduły
  \item[ {~\ref{ch:zachowania}. \hyperref[ch:zachowania]{Zachowania} -} ]zawiera opis budowy oraz działania zachowań złożonych\\robota wraz z przykładami
  \item[ {~\ref{ch:komunikacja}. \hyperref[ch:komunikacja]{Komunikacja} -} ]opisuje szczegóły implementacji warstwy połączeniowej, bazującej na interfejsie gniazd sieciowych i protokole UDP, a także komunikacji wewnętrznej bazującej na zdarzeniach
  \item[ {~\ref{ch:testowanie}. \hyperref[ch:testowanie]{Testowanie aplikacji} -} ]zawiera przebieg przykładowych testów i ich\\rezultaty
  \item[ {~\ref{ch:podsumowanie}. \hyperref[ch:podsumowanie]{Podsumowanie} -} ]zawiera wnioski końcowe z przebiegu projektowania\\aplikacji oraz przeprowadzonych testów.
\end{description}
