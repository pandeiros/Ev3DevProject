\chapter{Zachowania}
\label{ch:zachowania}

Zachowania definiują najwyższy stopień abstrakcji sterowania agentem. Oprócz właściwych akcji, które wykonują, mogą mieć zdefiniowane również zdarzenia, które wymuszają niestandardowy przebieg wykonania. Ponadto, zlecają również śledzenie konkretnych wartości na podłączonych urządzeniach.

\section{Budowa zachowań złożonych}

Każde zachowanie składa się z kilku elementów:
\begin{itemize}
    \item Listy stanów, z których każdy zawiera konkretną akcję
    \item Listy możliwych przejść dla danego stanu
    \item Reakcji, które wystąpią po zajściu danego zdarzenia (np. cofnięcie się po uderzeniu w przeszkodę).
\end{itemize}

Lista stanów, opisanych klasą {\tt RobotState}, zawiera zarówno stany podstawowe, opisujące zamierzone działanie wykonywanych akcji, ale także stany pośrednie - reakcje.

\subsection{Tworzenie zachowań}

Stworzenie zachowania może odbyć się na dwa sposoby. Pierwszym z nich jest skorzystanie z zachowania już zdefiniowanego w bibliotece. W tym przypadku zbierane są ustalone wcześniej, odpowiednio ukonkretnione akcje razem z warunkami przejść. Drugim wariantem jest stworzenie własnego zachowania. Jeśli taki typ zostanie wykryty, do agenta należy dostarczyć dodatkowe dane opisujące użyte akcje, przejścia i zdarzenia.

\ldots

% TODO XML ?

\subsection{Sterowanie wykonywanymi akcjami}

Klasa {\tt Behaviour} przechowuje informację o bieżącym stanie. W każdej iteracji pętli głównej robota, następuje sprawdzenie, czy nie otrzymano instrukcji zakończenia z warsty wyższej lub czy nie wystąpiło zdarzenie krytyczne. Następnie przetwarzany jest bieżący stan. Jeżeli akcja została wykonana, sprawdzany jest jej warunek zakończenia. W przypadku pomyślnego zakończenia akcji, następuje przejście do domyślnego stanu następnego. Jeśli natomiast w trakcie wykonywania akcji zostanie przechwycone zdarzenie, jest ono porównywane z listą obsługiwanych zdarzeń danego stanu. Typowy scenariusz zakłada wybranie stanu pośredniego, który prawidłowo zareaguje na zaistniałe zdarzenie, a następnie przekaże kontrolę dalej, zapewniając normalny dalszy przebieg (rysunek ...). Jeśli zdarzenie nie jest obsługiwane, ewentualna zmiana stanu jest ignorowana (zdarzenia krytyczne obsługiwane są przez warstwę wyższą).

% TODO Rysunek

W przypadku zaistnienia konieczności zmiany stanu, ustawiana jest specjalna flaga. Odczytuje ją klasa zachowania, pobierając następny stan i ustawiając go jako bieżący. Możliwe jest natomiast pominięcie definicji następnego stanu w przypadku akcji o nieokreślonych granicach wykonywania. Za poprawny przebieg wykonania odpowiadają wtedy zdefiniowane przejścia zachodzące po wystąpieniu zdarzenia lub zinterpretowane komunikaty z sieci.

% \subsection{Przejścia pomiędzy stanami}

% \section{Wykonywanie zachowania}

\subsection{Zbieranie danych}

Bieżące zachowanie robota może wymuszać zbieranie danych z podłączonych sensorów. Należy wtedy zdefiniować specjalną listę zawierającą typy sensorów, których pomiar ma być przekazywany dalej. W każdym kroku przetwarzania zachowania generowane są raporty typu SENSOR\_VALUE, które umieszczone w kolejce wiadomości robota mogą być przekazane innemu urządzeniu, domyślnie nadzorcy. Jest to rozszerzenie funkcjonalności zachowań o dodatkowy cel, bowiem np. znalezienie interesującego fragmentu otoczenia może skutkować zamianą logiki robota na inną z zamiarem eksploatacji danego miejsca (rysunek ...). Przykładowym scenariuszem może być np. szukanie powierzchni o konkretnym kolorze, a następie poruszanie się tylko w jej obrębie.

% TODO Rysunek, robot jeździ i znajduje kolorową powierchnię, a następnie zmienia cel na eksploatację.

\section{Przykładowe zachowania}

Definicje zachowań mogą przyjmować różne formy. Przykładowo, zadanie zwiedzanie otoczenia może być zdefiniowane jako poruszanie się z pilnowaniem lewej lub prawej krawędzi lub poruszanie się po prostej aż do napotkania przeszkody i obieranie ustalonego bądź losowego kąta obrotu. Niezależnie od zdefiniowanego rodzaju zachowania, ważny pozostaje jej cel. Jednakże jego wyznaczaniem i interpretacją zajmuje się warsta wyższa, a z poziomu zachowania można wydobyć tylko informację, czy dany cel został osiągnięty i czy nadal jest osiągalny. Poniżej przedstawione są przykładowe zachowania.

\subsection{Zwiedzanie otoczenia}

\subsection{Ruch po podłożu o zadanym kolorze}
