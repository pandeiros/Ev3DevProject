\chapter{Zachowania}
\label{ch:zachowania}

Zachowania definiują najwyższy stopień abstrakcji sterowania agentem. Oprócz właściwych akcji, które wykonują, mogą mieć zdefiniowane również zdarzenia, które wymuszają niestandardowy przebieg wykonania. Ponadto, zlecają również śledzenie konkretnych wartości na podłączonych urządzeniach.

\section{Budowa zachowań złożonych}

Każde zachowanie składa się z kilku elementów:
\begin{itemize}
    \item Listy stanów, z których każdy zawiera konkretną akcję oraz zdefiniowane przejście
    \item Reakcji, które wystąpią po zajściu danego zdarzenia (np. cofnięcie się po\,uderzeniu w przeszkodę)
    \item Stanu stopującego działanie agenta, wykorzystywanego w razie problemów
    \item Opcjonalnych parametrów.
\end{itemize}

\subsection{Tworzenie zachowań}

Stworzenie zachowania może odbyć się na dwa sposoby. Pierwszym z nich jest skorzystanie z zachowania już zdefiniowanego w bibliotece. W tym przypadku zbierane są ustalone wcześniej, odpowiednio ukonkretnione akcje razem z warunkami przejść. Drugim wariantem jest stworzenie własnego zachowania. Jeśli taki typ zostanie wykryty, do agenta należy dostarczyć dodatkowe dane opisujące użyte akcje, przejścia i zdarzenia.

% TODO XML ?

\subsection{Sterowanie wykonywanymi akcjami}

Klasa {\tt Behaviour} przechowuje informację o bieżącym stanie. W każdej iteracji pętli głównej robota, następuje sprawdzenie, czy nie otrzymano instrukcji zakończenia z warsty wyższej lub czy nie wystąpiło zdarzenie krytyczne. Następnie przetwarzany jest bieżący stan. Jeżeli akcja została wykonana, sprawdzany jest jej warunek zakończenia. W przypadku pomyślnego zakończenia akcji, następuje przejście do\,domyślnego stanu następnego. Jeśli natomiast w trakcie wykonywania akcji zostanie przechwycone zdarzenie, jest ono porównywane z listą obsługiwanych zdarzeń danego stanu. Typowy scenariusz zakłada wybranie stanu pośredniego, który prawidłowo zareaguje na zaistniałe zdarzenie, a następnie przekaże kontrolę dalej, zapewniając normalny dalszy przebieg (rysunek \ref{fig:behaviour_transitions}). Jeśli zdarzenie nie jest obsługiwane, ewentualna zmiana stanu jest ignorowana (zdarzenia krytyczne obsługiwane są przez warstwę wyższą).

\begin{figure}[!ht]
    \centering
        \includegraphics[width=0.75\textwidth]{diagrams/behaviour_transitions.png}
    \caption{Diagram stanów zachowania oraz przetwarzania reakcji.\label{fig:behaviour_transitions}}
\end{figure}

Jeśli stan zwróci wartość nieujemną, to oznacza ona identyfikator kolejnego stanu. Możliwe jest natomiast pominięcie definicji następnego stanu w przypadku akcji o nieokreślonych granicach wykonywania. Za poprawny przebieg wykonania odpowiadają wtedy zdefiniowane przejścia zachodzące po wystąpieniu zdarzenia lub zinterpretowane komunikaty z sieci.

\subsection{Zbieranie danych}

Bieżące zachowanie robota może wymuszać zbieranie danych z podłączonych sensorów. Należy wtedy zdefiniować specjalną listę zawierającą typy sensorów, których pomiar ma być przekazywany dalej. W każdym kroku przetwarzania zachowania generowane są raporty typu SENSOR\_VALUE, które umieszczone w kolejce wiadomości robota mogą być przekazane innemu urządzeniu, domyślnie nadzorcy. Jest\,to rozszerzenie funkcjonalności zachowań o dodatkowy cel, bowiem np. znalezienie interesującego fragmentu otoczenia może skutkować zamianą logiki robota na\,inną z zamiarem eksploatacji danego miejsca (rysunek \ref{fig:measurements}). Przykładowym scenariuszem może być np. szukanie powierzchni o konkretnym kolorze, a następie poruszanie się tylko w jej obrębie.

\begin{figure}[!ht]
    \centering
        \includegraphics[width=0.75\textwidth]{diagrams/measurements.png}
    \caption{Uproszczony schemat przykładowego przebiegu eksploracji z późniejszą eksploatacją.\label{fig:measurements}}
\end{figure}

\section{Przykładowe zachowania}

Definicje zachowań mogą przyjmować różne formy. Przykładowo, zadanie zwiedzania otoczenia może być zdefiniowane jako poruszanie się z pilnowaniem lewej lub prawej krawędzi lub poruszanie się po prostej aż do napotkania przeszkody i obieranie ustalonego bądź losowego kąta obrotu. Niezależnie od zdefiniowanego rodzaju zachowania, ważny pozostaje jej cel. Jednakże jego wyznaczaniem i interpretacją zajmuje się warsta wyższa, a z poziomu zachowania można wydobyć tylko informację, czy dany cel został osiągnięty i czy nadal jest osiągalny. Poniżej przedstawione jest przykładowe zachowanie.

\subsection{Zwiedzanie otoczenia}

Na rysunku \ref{fig:explore} znajduje się przykładowy algorytm zwiedzania otoczenia. Składa się z dwóch podstawowych stanów oraz dwóch stanów reakcji, zachodzących po zajściu konkretnych zdarzeń. Na uwagę zasługują specjalne stany pośrednie (oznaczone na diagramie nazwą \textit{pause}), których obecność jest krytyczna zarówno dla dokładności akcji, jak i pomiarów. Przez ich krótki czas trwania, robot wytraca swoją prędkość, kalibrując pozycję motorów oraz dokonując adekwatnego pomiaru z wybranych sensorów. Jeśli zaszło zdarzenie krytyczne lub zachowanie musi zostać zatrzymane, następuje przejście do stanu końcowego.

\begin{figure}[!ht]
    \centering
        \includegraphics[width=1\textwidth]{diagrams/explore.png}
    \caption{Szczegółowy diagram zwiedzania otoczenia.\label{fig:explore}}
\end{figure}

% TODO
% \subsection{Ruch po podłożu o zadanym kolorze}
