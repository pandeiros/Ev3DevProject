\section{Konfiguracja systemu}
\label{ch:konfiguracja_systemu}

W tym załączniku znajduje się instrukcja kompilacji aplikacji oraz sposobu uruchamiania.

\subsection{Kompilacja}

\textbf{Uwaga:} Kompilacja konfiguracji na architekturę typu ARM wymaga zainstalowanego programu {\tt arm-linux-gnueabi-g++}. Pozostałe dwie konfiguracje wymagają standardowego {\tt g++}.

\subsubsection{W środowisku graficznym}

Wgrane na płytę CD pliki zawierają projekt aplikacji dla środowiska NetBeans (użyta wersja: 8.0.2). Po wczytaniu projektu, na górze programu można wybrać z\,listy rozwijanej bieżącą konfigurację i ją skompilować (klawisz F11).

\subsubsection{W oknie konsoli}

Projekt aplikacji NetBeans tak naprawdę korzysta z zapisanych przez siebie plików Makefile. W katalogu głównym znajduje się plik ogólny Makefile, a w folderze {\tt nbproject} pliki poszczególnych konfiguracji. Proces budowania można rozpocząć poleceniem:
\begin{center}
    {\tt make <nazwa-konfiguracji>}\  lub\
    {\tt make all},
\end{center}
a usunięcie plików tymczasowych wykonuje komenda: {\tt make clean}

\subsection{Synchronizacja plików}

W celu szybkiego przesłanie plików na wiele robotów, przygotowany został specjalny katalog {\tt sync} wraz ze skryptem kopiującym. Edytując skrypt można podać następujące wartości parametrów:
\begin{itemize}
    \item Adresy ip wszystkich docelowych urządzeń, na które mają być wgrane pliki wykonywalne
    \item Nazwę użytkownika\textsuperscript{*}
    \item Katalog docelowy\textsuperscript{*}
    \item Hasło\textsuperscript{*}
\end{itemize}
\textsuperscript{*} - w przypadku, jeśli ten parametr jest identyczny na każdym urządzeniu.

Skrypt wykonuje kopię plików binarnych z katalogu {\tt dist} dla konfiguracji ARM i umieszcza ją w katalogu {\tt sync}. Następnie, przy pomocy polecenia {\tt rsync} z użyciem podanego hasła, zawartość katalogu zdalnego będzie dokładnym odwzorowaniem bieżącego katalogu.

\subsection{Uruchomienie aplikacji}

Aplikacja może być uruchomiona na dwa sposoby: z poziomu interfejsu graficznego klocka centralnego lub przez zdalne uruchomienie za pomocą połączenia ssh. Ten drugi wariant pozwala lepiej kontrolować komunikaty wyświetlane na ekranie, ale bardziej obciąża urządzenie. Z tego powodu nie powinno się uruchamiać wersji mobilnej z ustawieniami logowania na standardowe wyjście, ale jeśli jest to wymagane, należy ograniczyć rodzaj wyświetlanych wiadomości do minimum.

Przesłany plik wykonywalny można uruchomić w dwóch trybach: agent i master (domyśnie agent). Ponadto, aplikacja przyjmuje następujące opcjonalne argumenty wywołania:
\begin{itemize}
    \item model - rodzaj uruchomionego na robocie modelu (tylko w przypadku wariantu {\tt agent})
    \item port=<numer> - numer portu do komunikacji
    \item loglevel=<numer> - sprecyzowanie minimalnego poziomu logowania wiadomości, gdzie 0 to brak logowania, a 4 to poziom rozwlekły (verbose).
\end{itemize}

Zakończenie programu można przedwcześnie wymusić przez przesłanie sygnału SIGINT za pomocą kombinacji klawiszy CTRL+C.\\

%TODO Dodać obsługę touch sensora, w przypadku uruchomienia aplikacji nie z poziomu ssh.

Przykładowe uruchomienie:
\begin{center}
    {\tt ./ev3dev agent port=12233 model=A loglevel=2}
\end{center}
