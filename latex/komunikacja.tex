\chapter{Komunikacja}
\label{ch:komunikacja}

\section{Komunikacja przez sieć}

\section{Komunikacja wewnątrz jednostki}

\section{Zdarzenia}

\section{Utrata połączenia z nadzorcą}

Zdarzenie to może zajść w dwóch wariantach:

\subsection{Jeden agent traci połączenie z nadzorcą}

Robot przechodzi do stanu PANIC i przez adres rozgłoszeniowy informuje pozostałych agentów o utraconym połączeniu. Jeden z nich może odpowiedzieć pozytywnie, wysyłając potwierdzenie posiadania komunikacji z nadzorcą. Wtedy następuje synchronizacja dwóch agentów i ustalenie przekierowania wiadomości przez robota pośredniego (Rysunek ...). Jeżeli po pewnym czasie, nikt nie odpowie pozytywnie oraz nie zacznie się głosowanie opisane w punkcie poniżej, robot kończy swoje działanie.

% TODO Rysunek

\subsection{Wszyscy agenci tracą połączenie z nadzorcą}

Wszystkie roboty utraciły połączenie i żaden nie otrzymał odpowiedzi na zapytanie o przekierowanie. W takim przypadku odbywa się głosowanie. Każdy agent wysyła na adres rozgłoszeniowy swój osobisty wynik\footnote{Wynik agenta zależy m. in. od ilości przesłanych danych, poprawności wykonanych akcji czy czasu aktywności.}. Po pewnym czasie, kiedy wszystkie roboty znają już swoje wyniki, ten z najniższym rezultatem mianuje siebie nadzorcą. W przypadku tych samych wyników, o wyborze decyduje czynnik losowy. Wybierany zostaje najsłabszy rezultat, ponieważ warto pozwolić dobremu agentowi na kontynuację działań.

Problem jaki może napotkać grupa robotów, to jednoczesny wybór dwóch nadzorców. Ograniczenie użycia jednej wspólnej sieci nie eliminuje problemu gubienia pakietów, a przez to potencjalnej sytuacji, w której mamy wiele niezależnych głosowań, bądź głosowania są niepełne. W tym celu, oprócz wysłania przez danego agenta swoich danych, wysyła on dodatkowo liczbę oznaczającą ilość otrzymanych rekordów. W ten sposób agent A może dowiedzieć się od agenta B, że nie dostał informacji od innego agenta, np. C, kiedy połączenie między A i C jest niemożliwe. Ponadto, własny wynik jest ponownie rozsyłany za każdym zwiększeniem ilości informacji o pozostałych agentach. Jednostki nie posiadające kompletu informacji nie biorą udziału w finalnym głosowaniu, ponieważ mogą podjąć złą decyzję, a ponadto utrudniona komunikacja jest dla nich dodatkowo niekorzystna. Cały schemat działania opisuje Rysunek ... .

% TODO rysunek
